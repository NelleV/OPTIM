\documentclass{article}
%\usepackage[latin1]{inputenc}
\usepackage{graphicx,amssymb,amsmath,amsbsy,MnSymbol} % extensions pour maths
\usepackage{graphicx,mathenv}           % extensions pour figures
\usepackage[T1]{fontenc}        % pour les charactères accentués 
\usepackage[utf8]{inputenc} 
\usepackage{multicol}
\usepackage{wrapfig}
\usepackage{stmaryrd} % Pour les crochets d'ensemble d'entier
\usepackage{float}  % Pour placer les images là ou JE veux.

\DeclareMathOperator{\tr}{tr}
\DeclareMathOperator{\argmax}{argmax}
\DeclareMathOperator{\argmin}{argmin}
\DeclareMathOperator{\cov}{cov}


\setlength{\parindent}{0.0in}
\setlength{\parskip}{0.1in}
\setlength{\topmargin}{-0.4in}
\setlength{\topskip}{0.7in}    % between header and text
\setlength{\textheight}{9in} % height of main text
\setlength{\textwidth}{6in}    % width of text
\setlength{\oddsidemargin}{0in} % odd page left margin
\setlength{\evensidemargin}{0in} % even page left margin
%
%% Quelques raccourcis clavier :
\def\slantfrac#1#2{\kern.1em^{#1}\kern-.3em/\kern-.1em_{#2}}
\def\b#1{\mathbf{#1}}
\def\bs#1{\boldsymbol{#1}}
\def\m#1{\mathrm{#1}}
\bibliographystyle{acm}
%
\newcommand{\greeksym}[1]{{\usefont{U}{psy}{m}{n}#1}}
\newcommand{\inc}{\mbox{\small\greeksym{d}\hskip 0.05ex}}%
\pagenumbering{arabic}
\date{\today}
\title{Optimisation - DM1}
\author{Nelle Varoquaux}
\begin{document}
\maketitle

\section{Exercice 2.7}

Soit $a$ et $b$ dans $\mathbb{R}^n$.

\begin{align*}
\mathcal{H} & = & \{ x | \|x - a\|_2 \leq \|x - b\|_2 \} \\
	    & = & \{ x | (x - a)^T(x -a) \leq (x - b)^T(x -b) \} \\
	    & = & \{ x | (x^T - a^T)(x - a) \leq (x^T - b^T)(x -b) \} \\
	    & = & \{ x | (x^Tx - a^Tx - x^Ta + aa^T) \leq (x^Tx - b^Tx - x^Tb
	    + bb^T)  \} \\
	    & = & \{ x | (b^T - a^T)x + x^T(b - a) \leq  bb^T - aa^T \} \\
	    & = & \{ x | 2(b - a)^Tx \leq  bb^T - aa^T \} \\
	    & = & \{ x | (b - a)^Tx \leq  \frac{bb^T - aa^T}{2} \} \\
\end{align*}

$\mathcal{H}$ est donc un demi-espace.

\section{Exercice 2.35}

Posons $\mathcal{C} = \{ \mathbf{X} \in \mathcal{S}_n | \forall z \succeq 0,
z^T \mathbf{X} z
\leq 0\}$.

Il est évident que si $\mathbf{X} \in \mathcal{C}$, $\forall \theta \in
\mathbb{R}$, $\theta \mathbf{X} \in \mathcal{C}$. Donc $\mathcal{C}$ est un
cône.

De plus,

\begin{itemize}
\item $\mathcal{C}$ est convexe. Posons $\mathbf{X}_ 1, \mathbf{X_2} \in
  \mathcal{C}^2$. On a alors:
  \begin{itemize}
    \item $\forall z_1 \succeq 0, z_1^T \mathbf{X}_1 z_1 \geq 0$
    \item $\forall z_2 \succeq 0, z_2^T \mathbf{X}_2 z_2 \geq 0$
  \end{itemize}
  Donc, $\forall \theta \in \mathbb{R}$:
    \begin{itemize}
      \item $\theta z_1^T \mathbf{X}_1 z_1 \geq 0$
      \item $(1 - \theta) z_2^T \mathbf{X}_2 z_2 \geq 0$
    \end{itemize}
  Donc $\theta z_1^T \mathbf{X}_1 z_1  + (1 - \theta) z_2^T \mathbf{X}_2 z_2
  \geq 0$. De plus, $\theta z_1^T \mathbf{X}_1 z_1  + (1 - \theta) z_2^T
  \mathbf{X}_2 z_2 \in \mathcal{S}_n$. Donc, $\theta z_1^T \mathbf{X}_1 z_1  + (1 - \theta) z_2^T
  \mathbf{X}_2 z_2 \in \mathcal{C}$.

\item $\mathcal{C}$ est fermé, car son complémentaire est ouvert.
\item $\mathcal{C}$ est non-vide: $\mathbf{0} \in \mathcal{C}$
\item $\mathcal{C}$ est pointé: Soit $\mathbf{X} \in \mathcal{C}$. On a donc:
$\forall z \succeq 0, z^T \mathbf{X} z \geq 0$. On a alors:
\begin{itemize}
  \item $\forall z \succeq 0, z^T (- \mathbf{X}) z \leq 0$ par l'inégalité précédente.
  \item $\forall z \succeq 0, z^T (- \mathbf{X}) z \geq 0$, puisque $- \mathbf{X} \in
  \mathcal{C}$
\end{itemize}

Donc $\mathbf{X} = \mathbf{0}$
\end{itemize}

$\mathcal{C}$ est donc un cône propre.

\section{Exercice 2.36}

Posons $\mathcal{D}$ l'ensemble des matrices de distance euclidienne.

Montrons que $\mathcal{D}$ est convexe.

Soit $X, Y \in \mathcal{D}$.
On a par définition:
$\forall i, X_{ii} = Y_{ii} = 0$ et $\forall x$ tel que $\mathbf{1}^T x = 0$,
$x^T X x \leq 0$ et $x^T Y x \leq 0$.

On a donc, $\forall \theta \in \mathbb{R}$:
\begin{itemize}
\item $\theta X_{ii} = (1 - \theta) Y_{ii} = 0$
\item $\forall x$ tel que $\mathbf{1}^T x = 0$, $\theta x^T X x \leq 0$ et $(1
- \theta) x^T Y x \leq 0$, donc $\theta x^T X x + (1 - \theta) x^T Y x \leq
0$. Donc $x^T \( \theta X  + (1 - \theta) Y \) x \leq 0$. 
\end{itemize}
On a donc $\theta X + (1 - \theta) Y \in \mathcal{D}$. $\mathcal{D}$ est donc
convexe.

\section{Exercice 3.32}

\subsection{Question a}

Soit $f$ et $g$ convex, non décroissantes, et positives par intervalle. 

Supposons $f$ et $g$ deux fois différentiables.
On a alors: $f'' \geq 0$, $f' \geq 0$ et $f \geq 0$.
De même, $g'' \geq 0$, $g' \geq 0$ et $g \geq 0$

Posons $h = fg$. On a donc $h' = f'g + fg'$ et $h'' = f''g + 2g'f' + fg''$. On
a donc $h'' \geq 0$, et donc $h$ est convexe.

De même, si $f$ et $g$ sont non croissantes, on a alors: $f'' \geq 0$, $f'
\leq 0$ et $f \geq 0$ et $g'' \geq 0$, $g' \leq 0$ et $g \geq 0$. Et donc
$f'g' \geq 0$, ce qui implique $h'' \geq 0$, et donc $h$ convexe.

\subsection{Question b}

Soit $f$ concave, positive et non décroissante.
Supposons $f$ et $g$ deux fois différentiables.

On a alors: $f'' \leq 0$, $f' \geq 0$ et $f \geq 0$.
Soit $g$ concave, positive et non croissante. On a alors:  $g'' \leq 0$, $g'
\leq 0$ et $g \geq 0$.

Posons comme précédement, $h = fg$. On a donc $h' = f'g + fg'$ et $h'' = f''g
+ 2g'f' + fg''$. Or $f''g \leq 0$, $g'f' \leq 0$ et $fg'' \leq 0$. Donc, $h''
\leq 0$, $h$ est concave.

\subsection{Question c}

Soit $f$ convexe, non décroissant et positive.
Supposons $f$ et $g$ deux fois différentiables.

On a donc $f'' \geq 0$, $f'
\geq 0$ et $f \geq 0$.

Soit $g$ concave, non croissante et positive. On a donc $g'' \leq 0$, $g' \leq
0$ et $g \geq 0$.

Posons $h = \frac{f}{g}$. On a donc $h'= \frac{f'}{g} - \frac{fg'}{g^2}$ et 
$h''= \frac{f''g^2 - 2f'g'g + 2fg'^2 - fgg''}{g^3}$.

Le dénominatur de $h''$ est positif. De plus, $f''g^2 \geq 0$, $fg'g \leq 0$
donc $-fg'g^2 \geq 0$, $fg'^2 \geq 0$ et $fgg'' \leq 0$ donc $-fg''g \geq 0$.
On a donc $h'' \geq 0$ donc $h$ est convexe.

\section{Exercice 3.36}
\subsection{Question a}

Posons $f(x) = \max_{i = 1, \dots n} x_i = \| x \|_{\infty}$ sur $\mathbb{R}^n$.

On a donc, si $\| y \|_* \leq 1$, $f(y) = 0$ et $f(y) = \infty$ sinon. De
plus, $\| . \|_* = \| . \|_1$


\subsection{Question b}
Posons $f(x) = \sum_{i = 1}^r x_{\[i\]}$ sur $\mathbb{R}^n$.

\begin{align*}
y^Tx -f(x) & = & \sum_{i = 1}^n y_i x_i - \sum_{i = 1}^r x_i \\
	   & = &  \sum_{i = 1}^r (y_i - 1) x_i + \sum_{i = r + 1}^n y_i x_i
\end{align*}

Donc $y^Tx - f(x)$ n'est bornée que pour $y_{[i]} = 1$ et nulle ailleurs.
$f^*(y)$ est alors nulle sur ce domaine.

\subsection{Question c}
Posons $f(x) = \max_{i = 1, \dots, m} \(a_i x + b_i \)$ sur $\mathbb{R}$, avec
$a_1 \leq a_2 \leq \dots \leq a_m$.

\begin{align*}
y^Tx - f(x) & = & y x - \max_{i = 1, \dots, m} \(a_i x + b_i \)
\end{align*}

Si $m =1$, $y^Tx - f(x)$ n'est bornée que si $y = a_1$, et $f^*(x) = -b$, le
domaine de $f^*$ étant $a_1$.

\subsection{Question d}

Posons $f(x) = x^p$ sur $\mathbb{R}_{++}$, avec $p > 1$.

On a donc:
\begin{align*}
y^Tx - f(x) & = & y x - x^p
\end{align*}

$y^Tx - f(x)$ est bornée pour $y > 0$. Le maximum est atteint pour $x =
\(\frac{y}{p}\)^{\frac{1}{p - 1}}$. Donc $f^*(y) = y
\(\frac{y}{p}\)^{\frac{1}{p - 1}} + \(\frac{y}{p}\)^{\frac{p}{p - 1}}$

\end{document}
