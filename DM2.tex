\documentclass{article}
%\usepackage[latin1]{inputenc}
\usepackage{graphicx,amssymb,amsmath,amsbsy,MnSymbol} % extensions pour maths
\usepackage{graphicx,mathenv}           % extensions pour figures
\usepackage[T1]{fontenc}        % pour les charactères accentués 
\usepackage[utf8]{inputenc} 
\usepackage{multicol}
\usepackage{wrapfig}
\usepackage{stmaryrd} % Pour les crochets d'ensemble d'entier
\usepackage{float}  % Pour placer les images là ou JE veux.

\DeclareMathOperator{\tr}{tr}
\DeclareMathOperator{\argmax}{argmax}
\DeclareMathOperator{\argmin}{argmin}
\DeclareMathOperator{\cov}{cov}


\setlength{\parindent}{0.0in}
\setlength{\parskip}{0.1in}
\setlength{\topmargin}{-0.4in}
\setlength{\topskip}{0.7in}    % between header and text
\setlength{\textheight}{9in} % height of main text
\setlength{\textwidth}{6in}    % width of text
\setlength{\oddsidemargin}{0in} % odd page left margin
\setlength{\evensidemargin}{0in} % even page left margin
%
%% Quelques raccourcis clavier :
\def\slantfrac#1#2{\kern.1em^{#1}\kern-.3em/\kern-.1em_{#2}}
\def\b#1{\mathbf{#1}}
\def\bs#1{\boldsymbol{#1}}
\def\m#1{\mathrm{#1}}
\bibliographystyle{acm}
%
\newcommand{\greeksym}[1]{{\usefont{U}{psy}{m}{n}#1}}
\newcommand{\inc}{\mbox{\small\greeksym{d}\hskip 0.05ex}}%
\pagenumbering{arabic}
\date{\today}
\title{Optimisation - DM1}
\author{Nelle Varoquaux}
\begin{document}
\maketitle

\section{Exercice 5.5}

Considérons le problème suivant:

\begin{equation*}
\min_{G x \preccurlyeq 0; Ax = b} c^T x
\end{equation*}

Le lagrangien est alors:

\begin{align*}
\mathcal{L}(x, \lambda, \nu) & = & c^T+ \lambda^T (Gx - h) + \nu^T(Ax - b)
\end{align*}

On peut alors en déduire la fonction dual:

\begin{align*}
g(\lambda, \nu) & = & \inf_x \mathcal{L}(x, \lambda, \nu) \\
		& = & \inf_x c^T+ \lambda^T (Gx - h) + \nu^T(Ax - b)
\end{align*}

On a ici un problème linéaire en $x$.

\begin{equation*}
g(\lambda, \nu) = \begin{cases}
		  -b^T \nu & \mbox{si} c + G^T \lambda + A^T \nu = 0\\
		  - \infty & \mbox{sinon} 
		  \end{cases}
\end{equation*}
\section{Exercice 5.7}

\section{Exercice 5.9}

\section{Exercice 5.11}

\section{Exercice 5.17}


\end{document}
