\documentclass{article}
%\usepackage[latin1]{inputenc}
\usepackage{graphicx,amssymb,amsmath,amsbsy,MnSymbol} % extensions pour maths
\usepackage{graphicx,mathenv}           % extensions pour figures
\usepackage[T1]{fontenc}        % pour les charactères accentués 
\usepackage[utf8]{inputenc} 
\usepackage{multicol}
\usepackage{wrapfig}
\usepackage{stmaryrd} % Pour les crochets d'ensemble d'entier
\usepackage{float}  % Pour placer les images là ou JE veux.

\DeclareMathOperator{\tr}{tr}
\DeclareMathOperator{\argmax}{argmax}
\DeclareMathOperator{\argmin}{argmin}
\DeclareMathOperator{\cov}{cov}


\setlength{\parindent}{0.0in}
\setlength{\parskip}{0.1in}
\setlength{\topmargin}{-0.4in}
\setlength{\topskip}{0.7in}    % between header and text
\setlength{\textheight}{9in} % height of main text
\setlength{\textwidth}{6in}    % width of text
\setlength{\oddsidemargin}{0in} % odd page left margin
\setlength{\evensidemargin}{0in} % even page left margin
%
%% Quelques raccourcis clavier :
\def\slantfrac#1#2{\kern.1em^{#1}\kern-.3em/\kern-.1em_{#2}}
\def\b#1{\mathbf{#1}}
\def\bs#1{\boldsymbol{#1}}
\def\m#1{\mathrm{#1}}
\bibliographystyle{acm}
%
\newcommand{\greeksym}[1]{{\usefont{U}{psy}{m}{n}#1}}
\newcommand{\inc}{\mbox{\small\greeksym{d}\hskip 0.05ex}}%
\pagenumbering{arabic}
\date{\today}
\title{Optimisation - DM1}
\author{Nelle Varoquaux}
\begin{document}
\maketitle

\section{Exercice 5.5 - Dual of a general LP}

Considérons le problème suivant:

\begin{equation*}
\min_{G x \preccurlyeq 0; Ax = b} c^T x
\end{equation*}

Le lagrangien est alors:

\begin{align*}
\mathcal{L}(x, \lambda, \nu) & = & c^T+ \lambda^T (Gx - h) + \nu^T(Ax - b)
\end{align*}

On peut alors en déduire la fonction dual:

\begin{align*}
g(\lambda, \nu) & = & \inf_x \mathcal{L}(x, \lambda, \nu) \\
		& = & \inf_x c^T+ \lambda^T (Gx - h) + \nu^T(Ax - b)
\end{align*}

On a ici un problème linéaire en $x$.

\begin{equation*}
g(\lambda, \nu) = \begin{cases}
		  -b^T \nu & \mbox{si} c + G^T \lambda + A^T \nu = 0\\
		  - \infty & \mbox{sinon} 
		  \end{cases}
\end{equation*}

\section{Exercice 5.7 - Piecewise linear minimization}

Considérons le problème suivant:

\begin{equation}
\label{pbm_57}
\min_x \max_{i = 1 \dots m} (a_i^Tx + b_i)
\end{equation}

avec $x \in \mathbb{R}^n$

\subsection{Question 1}

Ce problème est équivalent à

\begin{equation*}
\min_{a_i^Tx + b_i = y_i
} \max_{i = 1 \dots m} y_i\end{equation*}

On obtient le problème dual:

\begin{equation*}
\max b^T \nu
\end{equation*}
 avec les contraintes $\mathbf{1}^T \nu = 1$ et $a^T \nu = 0$ pour $i =
 1, \dots m$.

\subsection{Question 2}

Faisons l'hypothèse que les $a_i$ sont triés par ordre croissant, et donc:
$a_1 \leq a_2 \leq \dots \leq a_m$.

Le graphe de $f$ est linéaire par morceau, avec les points d'arrêt $\frac{b_i - b_{i
+ 1}}{a_{i + 1} - a_i}$


On a donc le dual:

$f^*(y) = - b_i - (b_{i + 1} - b_i) \frac{y - a_i}{a_{i + 1} - a_i}$

\subsection{Question 3}


Considérons maintenant le problème suivant:

\begin{equation}
\label{sub_pbm_57}
f_0(x) = \log \( \sum_{i = 1}^m \exp(a_i^T x + b_i\)
\end{equation}

Notons $p^*_{gp}$ la valeur optimale du problème (\ref{sub_pbm_57}) et
$p^*_{pwl}$ celle du problème (\ref{pbm_57}).

Le problème dual associé à (\ref{sub_pbm_57}) est:

\begin{equation*}
\max \{ b^T \nu - \sum_{i = 1}^m \nu_i \log \nu_i \}
\end{equation*}

avec les contraintes: $\mathbf{1}^T \nu = 1$ et $a^T \nu = 0$ et $\nu \succeq 0$

\begin{align*}
\max \{ b^T \nu - \sum_{i = 1}^m \nu_i \log \nu_i \} - \max b^T \nu & \leq & \max b^T \nu - \min \{ \sum_{i = 1}^m \nu_i \log \nu_i \} - \max b^T \nu \\
& \leq & - \min \{ \sum_{i = 1}^m \nu_i \log \nu_i \}
\end{align*}

Grâce aux contraintes


\section{Exercice 5.9}

\section{Exercice 5.11}

Considérons le problème suivant:

\begin{equation}
\label{pbm_511}
\min \sum_{i = 1}^N \| A_i x + b_i \|_2 + \frac{1}{2} \| x - x_0 \|_2^2
\end{equation}

Introduisons les variables suivantes:

\begin{align*}
y_i & = & A_i x + b_i & i = 1, \dots, N \\
y_0 & = & x - x_0 & \\
\end{align*}

Le problème (\ref{pbm_511}) est composé de somme positive. On a donc:

\begin{align*}
(1) & = & \min \sum_{i = 1}^N \| A_i x + b_i \|_2 + \frac{1}{2} \| x - x_0
\|_2^2\\
    & = & \sum_{i = 1}^ N \min \| A_i x + b_i \|_2 + \min \frac {1}{2} \| x -
    x_0  \|_2^2
\end{align*}

On peut donc écrire le problème dual équivalent:

\begin{equation*}
\max \sum_{i = 1}^N b_i^T \nu_i - x_0 \nu_0 - \frac{1}{2} \| \nu_0\|_*^2
\end{equation*}

avec:

\begin{align*}
A_i^T \nu_i & = & 0 & i=1, \dots, m \\
\| nu_i \|_* & < & 1 & \\
\mathbf{1}^T \nu_0 & = & 0 &
\end{align*}

\section{Exercice 5.17 - Robust linear programming with polyhedral uncertainty}

Considerons le problème:

\begin{equation}
\label{pbm_517}
\min c^Tx
\end{equation}

avec $x$ tel que:

\begin{equation*}
\sup_{a \in \mathcal{P}_i} a^Tx \leq b_i, i = 1, \dots m
\end{equation*}

et $\mathcal{P}_i =  \{a | C_ia \preceq  d_i \}$

Étudions le problème annexe $\max_{a_i \in \mathcal{P}_i} a_i^T x$ avec la
contrainte $C_i x \preceq d_i$.
Il est équivalent à résoudre $\min_{a_i \in \mathcal{P}_i} - x^T a_i$, avec la
même contrainte.

\begin{align*}
\mathcal{L}(a_i, \lambda, \nu) & = & -x^T a_i + \lambda^T(C_i a_i - d_i)
\end{align*}

On a donc:

\begin{equation*}
g(\lambda, \nu) = \begin{cases}
		    - \lambda^T d & \mbox{si} \lambda^T C - a^T \\
		    - \infty & \mbox{sinon}
		  \end{cases}
\end{equation*}

En posant alors $z_i = - \lambda$, on obtient alors:

\begin{align*}
d_i^T z_i & \leq b_i \\
z_i^T C_i & = a_i^T \\
z_i & \preceq 0
\end{align*}

Le problème (\ref{pbm_517}) est donc équivalent au problème suivant:

\begin{equation*}
\min c^T x
\end{equation*}

avec les contraintes: $d_i^T z_i \leq b_i$, $z_i^T C_i = a_i^T$, $z_i \preceq 0$
\end{document}
