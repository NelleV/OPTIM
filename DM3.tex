\documentclass{article}
%\usepackage[latin1]{inputenc}
\usepackage{graphicx,amssymb,amsmath,amsbsy,MnSymbol} % extensions pour maths
\usepackage{graphicx,mathenv}           % extensions pour figures
\usepackage[T1]{fontenc}        % pour les charactères accentués 
\usepackage[utf8]{inputenc} 
\usepackage{multicol}
\usepackage{wrapfig}
\usepackage{stmaryrd} % Pour les crochets d'ensemble d'entier
\usepackage{float}  % Pour placer les images là ou JE veux.

\DeclareMathOperator{\tr}{tr}
\DeclareMathOperator{\argmax}{argmax}
\DeclareMathOperator{\argmin}{argmin}
\DeclareMathOperator{\cov}{cov}


\setlength{\parindent}{0.0in}
\setlength{\parskip}{0.1in}
\setlength{\topmargin}{-0.4in}
\setlength{\topskip}{0.7in}    % between header and text
\setlength{\textheight}{9in} % height of main text
\setlength{\textwidth}{6in}    % width of text
\setlength{\oddsidemargin}{0in} % odd page left margin
\setlength{\evensidemargin}{0in} % even page left margin
%
%% Quelques raccourcis clavier :
\def\slantfrac#1#2{\kern.1em^{#1}\kern-.3em/\kern-.1em_{#2}}
\def\b#1{\mathbf{#1}}
\def\bs#1{\boldsymbol{#1}}
\def\m#1{\mathrm{#1}}
\bibliographystyle{acm}
%
\newcommand{\greeksym}[1]{{\usefont{U}{psy}{m}{n}#1}}
\newcommand{\inc}{\mbox{\small\greeksym{d}\hskip 0.05ex}}%
\pagenumbering{arabic}
\date{\today}
\title{Optimisation - DM2}
\author{Nelle Varoquaux}
\begin{document}
\maketitle

\section{Exercice 1}

Soit $A \in \mathbb{R}^{m \times n}$, $b \in \mathbb{R}^m$ et le problème:

\begin{align}
\min_{x \in \mathbb{R}^n} f(x) = - \sum_{i = 1}^{m}\log(b_i - a_i^T x)
\end{align}


On a donc:

\begin{align*}
\nabla f(x) & = & - \sum_{i = 1}^m \frac{a_i}{b_i - a_i^T x} \\
\nabla^2 f(x) & = & - \sum_{i = 1}^m \frac{a_i^2}{(b_i - a_i^T x)^2} \\
\end{align*}

Dans le fichier ex1.py se trouve les fonctions correspondant au calcul du
gradient (dans un cas un peu plus général), de la hessienne, et de la méthode
de Newton.

\begin{figure}
\begin{center}
\includegraphics[width=300px]{hw3_convergences.png}
\end{center}
\end{figure}

\section{Exercice 2}

Étudions maintenant le problème suivant:

\begin{align*}
\min c^Tx
\end{align*}

avec les contraintes:

\begin{align*}
Ax \leq b
\end{align*}

Le code du fichier ex2.py contient un script, résolvant ce problème.

\end{document}
